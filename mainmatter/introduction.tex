%\newcommand{\itab}[1]{\hspace{15mm}\ilap{#1}}
\chapter{Introduction}
\label{ch:intro}

The study of patterns is an innate human ability to create abstract representations for better perception, understanding, and interpretation of the world.  Representations have no limitations on simplicity or complexity of the information they are trying to convey and offer a way identify relationships among phenomena.  In this thesis, we focus on the hidden patterns within somatic mutations and medical images of cancerous lesions, separately and jointly.  The patterns exploited in this case offer an insight how a patient can be characterized by relationships among patterns rather than in isolation for better personalized treatment.  

In current clinical practice, the gold standard for tumor analysis is an invasive biopsy and molecular profiling, but the tumor specimen is not an effective representative of the entire tumor due to the underlying spatial and temporal pathologic heterogeneity, underestimating the mutational landscape of a tumor.  The efficacy of personalized treatment is therefore limited by intratumor genetic heterogeneity since treatments only target specific molecular aberrations.  As medical imaging is omnipresent at initial and follow upstages of patient treatment, noninvasively visualizing a cancer’s appearance, such as intratumor heterogeneity on a macroscopic level allows clinicians to define the imaging phenotype of a cancer and integrate the functional, molecular, and metabolic information segment patient populations into phenotypic subsets that share similar prognoses and are likely to respond to similar therapies.  With increasing imaging and computational tools, it is only natural to hypothesize, can imaging phenotypes be parlyed into optimizing patient diagnosis.  In the context of patterns, the classical example is tumour staging in which the involvement of adjacent tissues, regional lymph nodes and distant metastasis has an immediate and important impact on prognostic stratification and on the choice of therapeutic options. Improved staging with the use of imaging might help in avoiding unnecessary surgical interventions, which increases the morbidity of disease without a substantial benefit in the reduction of mortality. On the other hand upstaging can lead to the choice of more aggressive therapies, which may reduce the recurrence rate.  The  



Why do we focus on somatic mutations and not the other abundant -omics ?  The lack of efficacy of treating the cancer comes from the heterogeneous accumulation of somatic mutations that result in a various subtypes of cancer.  Moreover, a distinct cancer is also heterogenous as the collection of mutations varies from patient to patient. For example, acute lymphoblastic leukemia (ALL) has 98\% remission rate in children in comparison to 25\% to 50\% in adults with the same treatment plans.  As such, there is a push to design treatments based on the molecular biology of the patient, by targeting genes that play a role in tumor angiogeneis.  For example, epidermal growth factor receptor (EGFR) is a cell surface receptor that plays a critical role in cell proliferatin, survival, and differentiation and is tightly regulated by various extracelluar ligands in healthy cells.  Deleterious mutations in the EGFR gene [B. Rude Voldborg, EGFR] cause the inactivation of the extracelluar domain, rendering this domain constitutively active even in the absence of upstream events such as ligand-binding or dimerization.  Targeted treatments of inhibiting EGFR has shown great promise in lung, kidney, and corectal cancers, but unfortuantely only a fraction are responders and some are resistant or acquire resistitance to the therapy.  These resistances occur as part of the functional redundancy that is inherent in molecular biology where multiple genes play a role in the signaling pathway.  In the case of the EGFR gene, mutually exclusively mutations in KRAS confer resistance to to EGFR-targeted treatment.  Thus methods to identify co-occurance of mutations can aid in identifying patients that will respond to the treatment, as well as identifying potential mutations for targeted treatment.  The sparse and high dimensionality of somatic mutation data, however, precludes us from using simple statistical models.  Considering a binary representation for the dataset, in ALL alone, there are at least 5000 somatic mutations, therefore $2^{n}$ permutations are possible.  Naively one can increase the number of parameters to fit a linear model, but that becomes computational intractable.  Before we explore the inference of images to somatic mutations, we consider the difficulty of learning complex underlying structure of mutations and models that can compress the data into a set of co-occurring mutations.  Specifically, we address how Bayesian nonparametric methods can be used to provide a flexible and computationally efficient structure for learning and inference of these complex systems.  




