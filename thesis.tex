%%% For print copies
%% set 'singlespace' option to set entire thesis to single space, and define "\printmode" to remove all hyperlinks for printed copies of the thesis. Delete all output files before changing this mode -- it will turn hyperref package on and off
%\documentclass[12pt,lot, lof, singlespace]{puthesis}
%\newcommand{\printmode}{}

%%% For the electronic copy, use doublespacing, define "\proquestmode" to use outlined links, instead of colored links. 
\documentclass[12pt,lot, lof]{puthesis}
\newcommand{\proquestmode}{}
% I prefer proquestmode to be off for electronic copies for normal use, since the colored links are less distracting. However when printed in black and white, the colored links are difficult to read. 

%%% For early drafts without some of the frontmatter
% Also see the "ifodd" command below to disable more frontmatter
%\documentclass[12pt]{puthesis}

%%%%%%%%%%%%%%%%%%%%%%%%%%%%%%%%%%%%%%%%%%%%%%%%%%%%%%%%%%%%%\
%%%% Author & title page info

\title{Patterns of Somatic Mutations and Medical Imaging in Human Cancer for Personalized Treatment}

\submitted{December 2019}  % degree conferral date (January, April, June, September, or November)
\copyrightyear{2019}  % year in which the copyright is secured by publication of the dissertation.
\author{Rahul Mehta}
\adviser{Muge Karaman}  %replace with the full name of your adviser
%\departmentprefix{Program in}  % defaults to "Department of", but programs need to change this.
\committee{Prof. Firstname1 Lastname1, Chair and Advisor \\ 
Prof. Muge Karaman  \\ 
Prof. Yang Dai \\ 
Prof. Jie Liang \\ 
Prof. Yang Lu}
\department{Bioengineering}

%%%%%%%%%%%%%%%%%%%%%%%%%%%%%%%%%%%%%%%%%%%%%%%%%%%%%%%%%%%%%\
%%%% Tweak float placements
% From: http://mintaka.sdsu.edu/GF/bibliog/latex/floats.html "Controlling LaTeX Floats"
% and based on: http://www.tex.ac.uk/cgi-bin/texfaq2html?label=floats
% LaTeX defaults listed at: http://people.cs.uu.nl/piet/floats/node1.html

% Alter some LaTeX defaults for better treatment of figures:
    % See p.105 of "TeX Unbound" for suggested values.
    % See pp. 199-200 of Lamport's "LaTeX" book for details.
    %   General parameters, for ALL pages:
    \renewcommand{\topfraction}{0.85}	% max fraction of floats at top
    \renewcommand{\bottomfraction}{0.6}	% max fraction of floats at bottom
    %   Parameters for TEXT pages (not float pages):
    \setcounter{topnumber}{2}
    \setcounter{bottomnumber}{2}
    \setcounter{totalnumber}{4}     % 2 may work better
    \setcounter{dbltopnumber}{2}    % for 2-column pages
    \renewcommand{\dbltopfraction}{0.66}	% fit big float above 2-col. text
    \renewcommand{\textfraction}{0.15}	% allow minimal text w. figs
    %   Parameters for FLOAT pages (not text pages):
    \renewcommand{\floatpagefraction}{0.66}	% require fuller float pages
	% N.B.: floatpagefraction MUST be less than topfraction !!
    \renewcommand{\dblfloatpagefraction}{0.66}	% require fuller float pages

% The documentclass already sets parameters to make a high penalty for widows and orphans. 

%%%%%%%%%%%%%%%%%%%%%%%%%%%%%%%%%%%%%%%%%%%%%%%%%%%%%%%%%%%%%\
%%%% Use packages
\usepackage{indentfirst}
\usepackage{amsfonts}

%%% For figures
\usepackage{graphicx}
%\usepackage{subfig,rotate}

%%% for comments
\usepackage{verbatim}

%%% For tables
\usepackage{multirow}
% Longtable lets you have tables that span multiple pages.
\usepackage{longtable}

% Booktabs produces far nicer tables than the standard LaTeX tables.
%   see: http://en.wikibooks.org/wiki/LaTeX/Tables
\usepackage{booktabs}

%set parameters for longtable:
% default caption width is 4in for longtable, but wider for normal tables
\setlength{\LTcapwidth}{\textwidth}



%%%%%%%%%%%%%%%%%%%%%%%%%%%%%%%%%%%%%%%%%%%%%%%%%%%%%%%%%%
%%% Printed vs. online formatting
\ifdefined\printmode

% Printed copy
% url package understands urls (with proper line-breaks) without hyperlinking them
\usepackage{url}


\else

\ifdefined\proquestmode
%ProQuest copy -- http://www.princeton.edu/~mudd/thesis/Submissionguide.pdf

% ProQuest requires a double spaced version (set previously). They will take an electronic copy, so we want links in the pdf, but also copies may be printed or made into microfilm in black and white, so we want outlined links instead of colored links.
\usepackage{hyperref}
\hypersetup{bookmarksnumbered}

% copy the already-set title and author to use in the pdf properties
\makeatletter
\hypersetup{pdftitle=\@title,pdfauthor=\@author}
\makeatother

\else
% Online copy

% adds internal linked references, pdf bookmarks, etc

% turn all references and citations into hyperlinks:
%  -- not for printed copies
% -- automatically includes url package
% options:
%   colorlinks makes links by coloring the text instead of putting a rectangle around the text.
\usepackage{hyperref}
\hypersetup{colorlinks,bookmarksnumbered}

% copy the already-set title and author to use in the pdf properties
\makeatletter
\hypersetup{pdftitle=\@title,pdfauthor=\@author}
\makeatother

% make the page number rather than the text be the link for ToC entries
%\hypersetup{linktocpage}
\fi % proquest or online formatting
\fi % printed or online formatting


%%%%%%%%%%%%%%%%%%%%%%%%%%%%%%%%%%%%%%%%%%%%%%%%%%%%%%%%%%%%%\
%%%% Define commands

% Define any custom commands that you want to use.
% For example, highlight notes for future edits to the thesis
%\newcommand{\todo}[1]{\textbf{\emph{TODO:}#1}}


% create an environment that will indent text
% see: http://latex.computersci.org/Reference/ListEnvironments
% 	\raggedright makes them left aligned instead of justified
\newenvironment{indenttext}{
    \begin{list}{}{ \itemsep 0in \itemindent 0in
    \labelsep 0in \labelwidth 0in
    \listparindent 0in
    \topsep 0in \partopsep 0in \parskip 0in \parsep 0in
    \leftmargin 1em \rightmargin 0in
    \raggedright
    }
    \item
  }
  {\end{list}}

% another environment that's an indented list, with no spaces between items -- if we want multiple items/lines. Useful in tables. Use \item inside the environment.
% 	\raggedright makes them left aligned instead of justified
\newenvironment{indentlist}{
    \begin{list}{}{ \itemsep 0in \itemindent 0in
    \labelsep 0in \labelwidth 0in
    \listparindent 0in
    \topsep 0in \partopsep 0in \parskip 0in \parsep 0in
    \leftmargin 1em \rightmargin 0in
    \raggedright
    }

  }
  {\end{list}}



%%%%%%%%%%%%%%%%%%%%%%%%%%%%%%%%%%%%%%%%%%%%%%%%%%%%%%%%%%%%%\
%%%% Front-matter

% For early drafts, you may want to disable some of the frontmatter. Simply change this to "\ifodd 1" to do so.
\ifodd 0
% front-matter disabled while writing chapters
\renewcommand{\maketitlepage}{}
\renewcommand*{\makecopyrightpage}{}
\renewcommand*{\makeabstract}{}

% you can just skip the \acknowledgements and \dedication commands to leave out these sections.

\else


\abstract{
Somatic mutations of cancerous lesions and their corresponding medical images offer endogenous and exogenous insights into tumor etiology for patient diagnosis and prognosis.  The confluence of mutational and imaging data has led to a new branch of informatics called imaging genomics which correlates the datasets for downstream tasks such as inference or prediction of mutations from a lesion image.  To account for the complexity and high dimensionality of both datasets, many models focus on a subset somatic mutations based on frequency, resulting in the common learning problem of overfitting, which is a cause of underutilizing the full dataset.  Therefore the models typically are only applicable to a small population of patients.  In this thesis, we exploit the pairing of Bayesian nonparametrics and inference networks to create a learning algorithm that translates imaging data into predictions of a patient's somatic mutation profile.

We start by considering the sparse phenomena of somatic mutation datasets, where there are many, few frequently occurring mutations and a long-tail of rare mutations.  Our goal is to decompose the dataset into interpretable features that are also biologically significant.  The standard algorithms of clustering or decomposition, such as Principal Component Analysis (PCA), of the data follows similar methodologies as seen in natural language process, but the same assumptions cannot be applied to biological systems.  A clustering approach assigns a mutation to a single cluster, however, a single mutation is known to belong in multiple distinct biological processes during tumor angiogenesis.  Analysis that decompose the data onto a smaller subspace assume the data is generated through a multinomial likelihood or by penalizing the likelihood using a penalty function such as elastic net for modeling sparsity.  A suitable method for analysis of somatic mutation datasets should explicitly model highly over-dispersed count data along with biological correlations of the mutations.  Through the use of beta-bernoulli stochastic process (BeBP) one can examine an unbounded number of features that are not distinct in the set of mutations.  Our work shows robust learning of interpretable features of somatic mutation datasets with respect to the full mutational profile.  We demonstrate our work on the Pan Cancer Somatic Mutation Dataset, validating the interpretable features by linking them to biologically significance using over-representation analysis.  The underlying features can then be used for understanding the biology of the lesion. 

Before we explore the common subspace of somatic mutations and imaging, we aim to identify local patterns in lesion images for prediction of the histology of breast lesions and patient treatment effect.  Before, the rise of deep neural networks many images relied on hand-crafted features that exploited the texture and shape information available in pixels, however, these features cannot always be applied in medical imaging domain as there is low variance between pixels as in the case of Positron emission tomography–computed tomography (PET-CT) or they are not based on RGB values like in Diffusion Weighted Magnetic Resonance Imaging (DW-MRI).  Moreover, although a deep learning approach is ideal due to it's data-driven nature, the need for a large dataset limits it's use in some medical domains.  Our work studies imaging features from two different imaging modalities to determine the influence of imaging features on prediction scores.  For DW-MRI we study histogram features based on the parameter maps generated from two exponential functions, Continuous Time Random Walk (CTRW) and Intravoxel Incoherent Motion (IVIM), generated from high \textit{b}-values and low \textit{b}-values respectively.  We show that a combination of features from different parameter maps is superior to modeling the underlying lesion histology.  We then show the value of modeling a lesion in the 3D space as the prediction of treatment response is influenced by the intra-heterogenity of a lesion and as such a volumetric representation offers a comprehensive profile of the lesion.  Though 2D lesion images are the most common when performing prediction tasks for malignancy, 3D representation offer a richer image feature set to exploit for complex prediction tasks

The central topic of our thesis is the correspondence between somatic mutations and medical image of lesions.  We build upon our previous work to help solve two issues in creating this model: the exact location of the lesion biopsy is unknown therefore we create a volumetric representation of the lesion, and the ability to account for the sparse and high dimensional nature of somatic mutation datasets.  To account for increasing modeling complexity we rely on the inference network of variational autoencoders (VAE).  Of course, as mentioned to full utilize VAEs, a larger dataset is needed, as such we build a pan cancer dataset of 1063 samples that compromises of point cloud representations of lesions and their corresponding somatic mutations.  This dataset is a specific subset of the already amazing work done in the collections of data from The Cancer Genomic Archive (TCGA) and The Cancer Imaging Archive (TCIA).  We demonstrate our model on this subset of data for predicting a patient's full somatic mutation profile on just the lesion images.    
}


\fi  % disable frontmatter


%%%%%%%%%%%%%%%%%%%%%%%%%%%%%%%%%%%%%%%%%%%%%%%%%%%%%%%%%%%%%\
%%%% Hide some chapters

%%% If you want to produce a pdf that includes only certain chapters, specify them with includeonly, in addition to including all chapters below.
%\includeonly{ch-intro/chapter-intro}
%%% You can also specify multiple chapters.
%\includeonly{ch-intro/chapter-intro,ch-usage/chapter-usage}
%\includeonly{chap1,chap2,chap3}


%%%%%%%%%%%%%%%%%%%%%%%%%%%%%%%%%%%%%%%%%%%%%%%%%%%%%%%%%%%%%
%%%% Notes:

% Footnotes should be placed after punctuation.\footnote{place here.}
% Generally, place citations before the period~\cite{anotherauthor}.
% The proper usage for i.e., and e.g., include commas ``(e.g., option A, option B)''

%%%%%%%%%%%%%%%%%%%%%%%%%%%%%%%%%%%%%%%%%%%%%%%%%%%%%%%%%%%%%
%%%% Import chapters

\begin{document}

\makefrontmatter

%\newcommand{\itab}[1]{\hspace{15mm}\ilap{#1}}
\chapter{Introduction}
\label{ch:intro}

The study of patterns is an innate human ability to create abstract representations for better perception, understanding, and interpretation of the world.  These patterns have no limitations on simplicity or complexity of the information they are trying to convey and offer a way to identify relationships among phenomena.  As datasets become increasingly complex and large, patterns become much more difficult to identify.  Human experts rely on computation tools to mimic human decision patterns via black-box models or simple linear regression tools.  In this thesis, we focus on imaging and genetic patterns, specifically dissecting complex relationships of somatic mutations and medical images of cancerous lesions, both separately and jointly.  By understanding the landscape of various patterns, we aim to improve precision medicine through early targeted treatment of cancer patients.  This has the potential to improve patient mortality rates, while simultaneously decreasing potent side effects of cancer treatments.

Medical imaging is omnipresent at initial and follow upstages of patient treatment to noninvasively visualize a cancer’s appearance.  In conjunction with patient phenotypic symptoms, clinicians can     To finalize a treatment plan, clinicians rely on molecular profiling via invasive biopsy, however, it not an effective representation of the entire tumor due to the underlying spatial and temporal pathological heterogeneity, underestimating the molecular landscape of a tumor.   The efficacy of personalized treatment is therefore limited by intratumor genetic heterogeneity since treatments only target specific molecular aberrations.  To help mitigate this issue, clinicians have defined imaging phenotypes of a cancer's appearance and integrate the functional, molecular, and metabolic information to segment patient populations into phenotypic subsets that share similar prognoses and are likely to respond to similar therapies.  In the context of patterns, the classical example is tumour staging in which the involvement of adjacent tissues, regional lymph nodes and distant metastasis has an immediate and important impact on prognostic stratification and on the choice of therapeutic options. With increasingly powerful imaging tools, it is only natural to hypothesize, can imaging phenotypes be parlayed into further optimizing patient diagnosis.  Improved staging with the use of imaging might help in avoiding unnecessary surgical interventions, which increases the morbidity of disease without a substantial benefit in the reduction of mortality. On the other hand upstaging can lead to the choice of more aggressive therapies, which may reduce the recurrence rate.    

Through the lens of computer vision, imaging phenotypes are analogous to imaging features, which are a set of quantitative measurements to identify an object, in this case, the object is a tumor.  Classical methods of comparing medical images rely on the textural and spatial relationships between voxels and then use machine learning models to discriminate between tumors for numerous tasks such as segmentation, diagnosis, and prognosis tasks [citations] in various imaging modalities.  The same features, however, lack the resolution to answer more complex questions or are not applicable for other imaging modalities.  For example, consider the problem of predicting treatment response of a patient.  In such a scenario the discriminatory power of texture imaging features between follow-up images is too small to determine treatment response.  Similarly, Diffusion Weighted Magnetic Resonance Imaging (DWI) is a powerful tool that represents the underlying heterogeneity of tissue, but the voxels are continuous values and not graycale pixels, so texture imaging featuresare not applicable.  Motivated by more complex questions and new imaging tools, we explore and identify different imaging features that capture a larger amount of information showing applications to Positron Emission Tomography - Computed Tomography (PET-CT) and DWI.  

As clinicians incorporate phenotype and genotypes during patient treatment planning, it is imperative they also have access to relevant genetic biomarkers.  Why do we focus on somatic mutations and not the other abundant -omics ?  The lack of efficacy of treating the cancer comes from the heterogeneous accumulation of somatic mutations that result in a various types of cancer.  Moreover, even in the same type of cancer, there is little homogeneity between patients.  For example, within the most common hematologic cancer, diffuse large-B cell lymphoma (DLBCL), systemic therapy cures the majority of patients; however, a significant minority will succumb to the disease.  As such, there is a push to design treatments based on the molecular biology of the patient, by targeting genes that play a role in tumor angiogeneis.  Unfortunately, cancer heterogeneity, also has profound implications for drug therapy in cancer as the targeted therapy does not take into account the possibility of a low-frequency subpopulation harbouring a resistance.  For example, epidermal growth factor receptor (EGFR) is a cell surface receptor that plays a critical role in cell proliferation, survival, and differentiation and is tightly regulated by various extracelluar ligands in healthy cells. Targeted treatments of inhibiting EGFR has shown great promise in lung, kidney, and colorectal cancers, but unfortunately only a fraction are responders and some are resistant or acquire resistance to the therapy.  These resistances occur as part of the functional redundancy that is inherent in molecular biology where multiple genes play a role in the signaling pathway.  In the case of the EGFR gene, mutually exclusively mutations in KRAS confer resistance to to EGFR-targeted treatment.  Thus methods to identify co-occurrence of mutations can aid in identifying patients that will respond to the treatment, as well as identifying potential mutations for targeted treatment.  The sparse and high dimensionality of somatic mutation data, however, precludes us from using simple statistical models.  Considering a binary representation for the dataset, in DLBCL alone, there are at least 5000 somatic mutations, therefore $2^{n}$ permutations are possible.  Naively one can increase the number of parameters to fit a linear model, but that becomes computational intractable.  We consider the difficulty of learning the complex underlying structure of mutations and create a model to describe co-occurring mutations regardless of their frequency using models based on the Beta-Bernoulli Process (BeBP).  These co-occurring mutations can then be used as genetic biomarkers and other downstream tasks such as further patient subtyping.

In the problems discussed so far, we have assumed the different modalities of cancer data are only studied concurrently.  However, as cancer is a result of a combination of genetic instabilities and epigentic alterations, many integration based models have become popular for analyzing genetic datasets.  For example, one can predict breast cancer in a patient by identifying DNA sequence variation through association studies in family- and population-based data, and then integrating it with molecular pathway information [citation].  Extensions to integrate imaging and genetic data has gained traction in a nascent topic called radiogenomics or imaging genomics, but the structure and distribution of data generated by different modalities makes it a very challenging task. Many studies have focused on prediction of a few frequent genetic markers or a large set of genetic markers from imaging features, which limits the applicability of models to very specific tasks. Our proposed method relates the set modalities follows the paradigm that is often seen image to text or text to image generation.  To account for the complexity and sparsity of in the medical and biology domain we utilize paired variational auto-encoders with conditional normalizing flow priors that encourages the datasets to share similar subsets of the large set of possible behaviors.  We can use this shared space to predict patients' mutation profile based on their corresponding medical image.  The benefits of this model are twofold: patient mortality is best reduced when the disease is targeted during its early stages where the mutational landscape within a tumor is still relatively homogeneous and cancer spread is limited and we may uncover interesting relationships between the two vastly different datasets.

\section{Thesis Organization}

We now provide an overview of the contributions of each chapter, including methodologies and results, as well as an overview of the chapter structure.  Since we address a number of different problems in subsequent chapters, we provide more through literature reviews of prior relevant work for each of the models.

\subsection{Background}

We begin by reviewing many of the imaging and statistical concepts that are utilized throughout this thesis. The chapter starts with an overview of current image feature generation techniques for two and three dimensional images.  This will serve as a backbone to the reasoning of why we choose specific imaging features for our tasks.

We then describe methods in probabilistic models and their use in modeling sparse high-dimensional data by presenting latent feature models. Together, these concepts enable examination of the importance of prior distributions, the distribution of the unknown parameters, and the posterior distribution, the updated parameters given the data.  As exact inference is infeasible for the relevant parameters, the chapter then moves onto Bayesian inference techniques specifically amortized inference and variational inference. We conclude the chapter with background material on the most common stochastic processes we use in developing our Bayesian nonparametric models: the Dirichlet process and the beta process.


\subsection{Contributions}

The original contributions of this thesis follow in the remaining chapters, where chapter 3 introduces 3D imaging features and histogram features for treatment response prediction and tumor prediction respectively.   Thorough empirical examination of datasets that incorporate images from PET/CT and DWI we show our model results in very competitive predictive and interpretative performance. In Chapter 4, we introduce a framework for determining correlated sets of co-occurring somatic mutations.  Using datasets from The Cancer Genomic Archive (TCGA) we show we can generate biologically interpretable co-occurring mutations for patient subtyping and targeted treatment.  In Chapter 5, we introduce a framework for mapping somatic mutations and 3D imaging features of tumors into the same subspace for predicting somatic mutations from a patient's corresponding tumor image.  To show predictive performance of our model, we create a dataset of 3D point cloud representations of tumors from a subset of the The Cancer Imaging Archive data that has corresponding somatic mutation.  


% If you've disabled frontmatter, you can insert the toc manually
%\tableofcontents\clearpage

% \include lets us split up the document (and each include starts a new page):

% Make the bibliography single spaced
\singlespacing
\bibliographystyle{plain}

% add the Bibliography to the Table of Contents
\cleardoublepage
\ifdefined\phantomsection
  \phantomsection  % makes hyperref recognize this section properly for pdf link
\else
\fi
\addcontentsline{toc}{chapter}{Bibliography}

% include your .bib file
%\bibliography{backmatter/references.bib}

\end{document}

